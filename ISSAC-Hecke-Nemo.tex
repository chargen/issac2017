\documentclass{sig-alternate-05-2015}
\usepackage{paralist}

\begin{document}
\setcopyright{acmcopyright}
\conferenceinfo{ISSAC 2017}{July 25--28, 2017, Kaiserslautern, Germany}

\newtheorem{alg}{Algorithm}
\newtheorem{definition}{Definition}
\newtheorem{Assertion}{Assertion}

\makeatletter
\def\Ddots{\mathinner{\mkern1mu\raise\p@
\vbox{\kern7\p@\hbox{.}}\mkern2mu
\raise4\p@\hbox{.}\mkern2mu\raise7\p@\hbox{.}\mkern1mu}}
\makeatother

\title{Hecke/Nemo: Number Theory packages for the Julia programming language.}

\numberofauthors{3}
\author{
\alignauthor Claus Fieker\\
   \affaddr{Technische Universit\"{a}t Kaiserslautern}\\
   \affaddr{Fachbereich Mathematik, Postfach 3049,}\\
   \affaddr{67653 Kaiserslautern, Germany}\\
   \email{fieker@mathematik.uni-kl.de}
\alignauthor William Hart\\
   \affaddr{Technische Universit\"{a}t Kaiserslautern}\\
   \affaddr{Fachbereich Mathematik, Postfach 3049,}\\
   \affaddr{67653 Kaiserslautern, Germany}\\
   \email{goodwillhart@googlemail.com}
\alignauthor Fredrik Johansson\\
   \affaddr{Inria Bordeaux}\\
   \affaddr{33400 Talence, France}\\
   \email{fredrik.johansson@gmail.com}
}

\maketitle

\begin{abstract}
We introduce two new packages, Hecke and Nemo, written in the Julia programming language
for basic arithmetic and number theory. We demonstrate that high performance generic
algorithms can be implemented in Julia, without the need to resort to a low-level C
implementation. We also describe the various Julia wrappers of existing C libraries
such as Flint, Arb and Antic. We give examples of how to use Hecke and Nemo and discuss
some of the algorithms that we have implemented to provide high performance basic
arithmetic.
\end{abstract}


\section{Introduction}
\begin{thebibliography}{99}

\bibitem{magma}
J. J. Cannon, W. Bosma (Eds.) {\em Handbook of Magma Functions}, Edition 2.13 (2006)

\bibitem{mca} 
J. von zur Gathen and J. Gerhard. {\em Modern Computer Algebra}. Cambridge University Press, 1999.

\bibitem{flint}
W. Hart {\em FLINT}, open-source C-library. \texttt{http://www.flintlib.org}

\bibitem{ntl}
V. Shoup {\em NTL}, open-source C++ library. \texttt{http://www.shoup.net/ntl/}

\bibitem{sage}
W. Stein {\em SAGE Mathematics Software}.  \texttt{http://www.sagemath.org}

\end{thebibliography}
\end{document}
